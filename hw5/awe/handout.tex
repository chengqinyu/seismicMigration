\author{Isaac Newton}
\title{GPGN658: Seismic migration}{AWE}

% ------------------------------------------------------------
In this homework, you will write code for time-domain acoustic
finite-differences modeling. This is one of the most basic programs
employed in seismic imaging. I used in class for this program the name
of ``Hello World!'' of seismic imaging. Everyone involved in seismic
imaging must write this program once in their life. This is what you
will do in this homework.

Your assignment is to modify an acoustic finite-differences modeling
program and compute wavefields and data recorded on the surface. You
will use the constant-density and the variable-density acoustic
wave-equations.

\textbf{This is an individual assignment and absolutely no
  collaboration on code is allowed}.
% ------------------------------------------------------------

% ------------------------------------------------------------
\inputdir{exercise}
% ------------------------------------------------------------
%\sideplot{wava}{width=\textwidth}{Source wavelet.}
%\sideplot{vp}{width=\textwidth}{Velocity model.}
%\sideplot{ro}{width=\textwidth}{Density model.}
%\sideplot{wa}{width=\textwidth}{Wavefield.}
%\sideplot{da}{width=\textwidth}{Data.}

% ------------------------------------------------------------
\section{Exercise}

\begin{enumerate}
\item The program \texttt{AFDM.c} implements time-domain
  finite-differences modeling for the constant-density acoustic
  wave-equation. Your task is to add the density term to this
  program. Refer to the course slides for details about what needs to
  be added and where. Add comments in the code to indicate your
  modifications.

\item Run \texttt{scons view} after your code is modified. All figures
  are rebuilt with your new code and displayed on screen.

\item Run \texttt{scons lock} once you are satisfied with your
  results. All figures are copied to the storage directory.

\item Add comments to this document indicating the changes to the
  simulated data and wavefields. Are your results expected? Describe
  the data and wavefield figures indicating what the various events
  represent.

\item \texttt{cd awe}, run \texttt{scons handout.read} to build your
  answer. A PDF file is constructed using your newly created figures
  and modifications to the text. The modified code is automatically
  added to the document.

\end{enumerate}

% ------------------------------------------------------------
\section{Wrap-up}

After you are satisfied that your document looks ok, print it from the
PDF viewer and bring it to class.

% ------------------------------------------------------------
\newpage
\section{AFDM.c}
\tiny
\lstinputlisting{exercise/AFDM.c}
\normalsize


