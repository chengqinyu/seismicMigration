\author{Xinming Wu}
\title{GPGN658: Seismic migration}{RTM}

% ------------------------------------------------------------
In this homework, you will use a finite-differences modeling code,
similar to the one you wrote in the preceding homework, to implement
basic reverse time migration. I do not expect you to be concerned with
the efficiency of your implementation at this time. This
implementation of reverse-time migration does not require that you
write any new C code. You will use pre-existing Madagascar programs,
but you will modify the \texttt{SConstruct} file to combine those
programs.

\textbf{This is an individual assignment and absolutely no
  collaboration on code is allowed.}
% ------------------------------------------------------------

% ------------------------------------------------------------
\inputdir{exercise}
% ------------------------------------------------------------
\sideplot{vo}{width=\textwidth}{Velocity.}
\sideplot{ra}{width=\textwidth}{Density.}
\sideplot{dr0}{width=\textwidth}{Data w/ direct arrival.}
\sideplot{dr1}{width=\textwidth}{Data w/o direct arrival.}

% ------------------------------------------------------------
\section{Exercise}

Using the finite-differences modeling function \texttt{awefd},
construct an image of the subsurface. This function takes the
following parameters: \\
\texttt{awefd(odat,owfl,idat,velo,dens,sou,rec,custom,par)}
\begin{itemize}
\item \texttt{odat}: output data $d\left( x,t \right)$
\item \texttt{owfl}: output wavefield $u \left( z,x,t \right)$
\item \texttt{idat}: input data (wavelet)
\item \texttt{velo}: velocity model $v \left( z,x \right)$
\item \texttt{dens}: density model $\rho \left( z,x \right)$
\item \texttt{sou}: source coordinates
\item \texttt{rec}: receiver coordinates
\item \texttt{custom}: custom parameters
\item\texttt{par}: parameter dictionary
\end{itemize}

Design an imaging procedure following the generic scheme developed in
class. Your task is to identify Madagascar programs necessary to
implement reverse-time migration in two different ways and generate
the appropriate \texttt{Flows} in the \texttt{SConstruct}. Explain in
detail how your imaging procedures work. 

\begin{enumerate}
\item Use your imaging procedure to generate images based on recorded
  data in Figures~\ref{fig:dr0} and \ref{fig:dr1}. For this exercise,
  use the constant density \texttt{rb.rsf} for imaging. Include those
  two images in this document. Are the images different from
  each-other? How? Why?
%\sideplot{rb}{width=\textwidth}{The constant density \texttt{rb.rsf}.}
\sideplot{imag0rbic0}{width=\textwidth}
{\textcolor{blue}{data in Figure~\ref{fig:dr0}}, \textcolor{green}{constant density }, 
and \textcolor{red}{ first I.C.}.}
\sideplot{imag1rbic0}{width=\textwidth}
{\textcolor{blue}{data in Figure~\ref{fig:dr1}}, \textcolor{green}{constant density }, 
and \textcolor{red}{ first I.C.}.}
\sideplot{imag0rbic1}{width=\textwidth}
{\textcolor{blue}{data in Figure~\ref{fig:dr0}}, \textcolor{green}{constant density }, 
and \textcolor{red}{ second I.C.}.}
\sideplot{imag1rbic1}{width=\textwidth}
{\textcolor{blue}{data in Figure~\ref{fig:dr1}}, \textcolor{green}{constant density }, 
and \textcolor{red}{ second I.C.}.}

\item Use your imaging procedure to generate images based on recorded
  data in Figures~\ref{fig:dr0} and \ref{fig:dr1}. For this exercise,
  use the variable density \texttt{ra.rsf} for imaging. Include those
  two images in this document. Are the images different from
  each-other? How? Why? How do your images compare with the ones from
  the preceding exercise?
\sideplot{imag0raic0}{width=\textwidth}
{\textcolor{blue}{data in Figure~\ref{fig:dr0}}, \textcolor{green}{variable density }, 
and \textcolor{red}{first I.C.}.}
\sideplot{imag1raic0}{width=\textwidth}
{\textcolor{blue}{data in Figure~\ref{fig:dr1}}, \textcolor{green}{variable density }, 
and \textcolor{red}{first I.C.}.}
\sideplot{imag0raic1}{width=\textwidth}
{\textcolor{blue}{data in Figure~\ref{fig:dr0}}, \textcolor{green}{variable density }, 
and \textcolor{red}{second I.C.}.}
\sideplot{imag1raic1}{width=\textwidth}
{\textcolor{blue}{data in Figure~\ref{fig:dr1}}, \textcolor{green}{variable density }, 
and \textcolor{red}{second I.C.}.}
\end{enumerate}
\section{}
\section{}
\section{}
\section{}
\section{}
\section{}
\section{Comments}
\subsection{Codes}
The block of codes with gray background in the attached \texttt{SConstruct} are
designed to perform reverse-time migration.
\subsection{Results}
\textbf{1) using constant density}\\
1a) Using two different implementations for the conventional image condition, I
obtain the same migrated images as shown in Figures 5 and 7, Figures 6 and 8.\\
1b) Using the recorded data shown in Figure 3:\\ 
In this data, the direct arrival is much stronger than the reflected arrivals, 
therefore, the interfaces in the migrated images (Figures 5 and 7) are much 
weaker than the low-frequency feature (at $x=1.5$ km and $z=0.0$ km) due to the 
strong direct arrivals.\\
1c) Using the recorded data shown in Figure 4:\\ 
In this data, we only have the reflected arrivals while the direct arrival is
removed, therefore, the interfaces are strong and obvious in the migrated images
(Figures 6 and 8) and we do not see the feature at at $x=1.5$ km and $z=0.0$ km. 
In Figures 6 and 8, we observe smile-shape artifacts at the ends of the two 
horizontal reflectors because of the truncation of the data. We also see some
artifacts at the corners at $x=1$ km because these corners are refractors.\\
\textbf{2) using variable density}\\
2a) Using two different implementations for the conventional image condition, I
obtain the same migrated images as shown in Figures 9 and 11, Figures 10 and 12.\\
2b) Using the recorded data shown in Figure 3:\\ 
I obtain almost the same migrated images (Figures 9 and 11) as the ones (Figures 5
and 7) with constant density.\\
2c) Using the recorded data shown in Figure 4:\\ 
I obtain the same horizontal reflectors and nearby artifacts in the migrated
images (Figures 10 and 12) as using the ones (Figures 6 and 8) with constant 
density. However, I can also see the low-frequency features extended from the
top to the horizontal reflectors because the consistence of reflected events
between the source wavefield and data wavefield always exists.


% ------------------------------------------------------------
\newpage
\section{SConstruct}
\tiny
\definecolor{keywords}{RGB}{255,0,90}
\definecolor{comments}{RGB}{0,0,113}
\definecolor{red}{RGB}{160,0,0}
\definecolor{green}{RGB}{0,150,0}
\definecolor{gray}{RGB}{168,168,168}
\lstset{numbers=left,language=Python, 
        basicstyle=\ttfamily\small, 
        keywordstyle=\color{keywords},
        commentstyle=\color{comments},
        stringstyle=\color{red},
        showstringspaces=false,
        aboveskip=0pt,belowskip=0pt,
        identifierstyle=\color{green}}
\lstinputlisting[firstline=1,lastline=90]{exercise/SConstruct}
\lstinputlisting[backgroundcolor=\color{gray},firstline=91,lastline=116,
firstnumber=91]{exercise/SConstruct}
\lstinputlisting[firstline=117,firstnumber=117,lastline=117]{exercise/SConstruct}
\normalsize


